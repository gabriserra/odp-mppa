\documentclass{trkalray}
\usepackage{listings}
\usepackage[toc,page]{appendix}
\usepackage{makeidx}
\usepackage{algorithm}
\usepackage{algorithmic}
\usepackage{caption}
\usepackage{graphicx}
\usepackage{listings}
\lstset{ %
frame=single
}

\newcommand{\MPPA}{MPPA\texttrademark\space}

\version{KETD-355}{W23}{2015}


\author{%
Kalray S.A.\autref{1}
}
\docowner{Nicolas Morey-Chaisemartin}{nmorey@kalray.eu}
\vers{1.1}
\institute{%
\autlabel{1} \email{support@kalray.eu},
Kalray S.A.
}

\abstract{%
This document describes how to use K1-ODP
}

\keywords{%
Dataflow, MPPAIPC, Examples
}

\renewcommand{\lstlistingname}{Code}% Listing -> Code


\makeindex

\title{K1-ODP Manual}

\begin{document}

\maketitle

\tableofcontents

\newpage
\section{Introduction}

This documents briefly describes how to use ODP for the MPPA platform.

ODP-MPPA is currently based on ODP 1.6.

Additional generic documentation can be found on ODP website and in
\begin{lstlisting}
/usr/local/k1tools/doc/ODP/mppa/
\end{lstlisting}

\section{Requirements}

Working with K1-ODP requires two sets of packages:
\begin{itemize}
\item[-]{A complete K1 Toolchain using the k1-tools package}
\item[-]{K1-ODP Library using the k1-odp package}
\end{itemize}

The K1-ODP files will be installed under the
\texttt{/usr/local/k1tools} directory.

\section{Choosing a version}
K1-ODP is available in multiple ports in the k1-odp package, one for
each of the platforms available for executing ODP applications.

In this version, two ports are available:
\begin{itemize}
\item[-]{\texttt{k1b-kalray-nodeos}}
\item[-]{\texttt{k1b-kalray-nodeos\_simu}}
\item[-]{\texttt{k1b-kalray-mos}}
\item[-]{\texttt{k1b-kalray-mos\_simu}}
\end{itemize}

\subsection{k1b-kalray-nodeos}

This port targets one Bostan MPPA cluster on hardware.

\subsection{k1b-kalray-mos}

Same target as \texttt{k1b-kalray-nodeos}, but running directly on
Kalray Hypervisor

\subsection{k1b-kalray-nodeos\_simu}

This ports targets one Bostan MPPA cluster in simulation.

Using the simulator, this ports allows simulated ODP applications to
have transparent access to the x86 network interfaces.

In the current version, this port supports these interfaces:
\begin{itemize}
\item[-]{loop}
\item[-]{magic:<if\_name>}
\end{itemize}
\subsection{k1b-kalray-mos\_simu}

Same target as \texttt{k1b-kalray-nodeos\_simu}, but running directly on
Kalray Hypervisor

\section{Platform Settings}
\subsection{Rx Threads}
ODP internally allocates 2 cores (Rx Threads) per cluster to convert the incoming
Ethernet, NoC, and PCIe trafic to ODP packets.

The number of Rx Threads can be selected at runtime from 1 to 6,
depdning on the application requirement. Each thread can handle
approximately 1.8Mpps. Thus application with low pps requirements can
save 1 core for computation and on the other hand, high pps
application should dedicate more cores to Rx Threads to avoid packet
drop.

The number of threads can be configured by providing a
\texttt{odp\_platform\_init\_t} structure to
\texttt{odp\_init\_global} and setting the \texttt{n\_rx\_thr} value.

\begin{lstlisting}
	odp_platform_init_t platform_params = { .n_rx_thr = 1 };

	if (odp_init_global(NULL, &platform_params)) {
		fprintf(stderr, "Error: ODP global init failed.\n");
		exit(EXIT_FAILURE);
	}
\end{lstlisting}

\section{Packet IO Interfaces}

K1-ODP provides multiple pktio types that can be used to communicate
with Ethernet ports, or clusters.

\begin{itemize}
\item[-]{\texttt{loop}: Software loopback interface}
\item[-]{\texttt{e<X>}: 40G Ethernet interface of IOETH \texttt{<X>}}
\item[-]{\texttt{e<X>p<Y>}: 1/10G Ethernet interface \texttt{<Y>} of
  IOETH \texttt{<X>}}
\item[-]{\texttt{cluster<X>}: Interface to cluster \texttt{<X>}}
\item[-]{\texttt{magic:<if>}: Link to the x86 interface
  \texttt{<if>}. This is only available on simu platforms}
\end{itemize}

\subsection{Customizing Ethernet interface behaviour}

It is possible to configure the way ODP handles Ethernet interfaces.
This is done by passing options through the pktio name:
\begin{lstlisting}
e0p0:<option1>:<option2>
\end{lstlisting}

\subsubsection{Rx resources}

By default Ethernet packets from a lane are distributed among 20 NoC Rx on each
cluster that opens this lane.
When working with 40G lane, or with high PPS situations, this is often
not sufficient to handle the high performance requirement.
To configure this value, the option tags can be used:
\begin{lstlisting}
e0:tags=120
\end{lstlisting}

Not that it is not possuble to use more than 120 tags per pktio nor
250 tags for all pktios.

\subsubsection{Jumbo}

Jumbo frames are not allowed by default. To enable them, the option
\texttt{jumbo} must be used.
Note that the option must be the same (on or off) for all the cluster
that use an Ethernet interface

\begin{lstlisting}
e0:jumbo
\end{lstlisting}

\subsubsection{Loopback}

It is possible to put an Ethernet lane in loopback mode. In this mode,
it is not required for a cable to be plugged in, nor for the board to
have a physical connector connected to the lane.
When loopback mode is enabled, all the packets sent to this lane by
the clusters are looped back into the HW Dispatcher and sent back to
the cluster that opened this pktio.

Note that the option must be the same (on or off) for all the cluster
that use an Ethernet interface

\begin{lstlisting}
e0:loop
\end{lstlisting}

\subsubsection{Dispatch}

User can apply dispatch rules to ethernet physical links. According to
these rules, input ethernet packets will be dispatched to specific
clusters. It allows for example to guarantee that all packets from an IP
address are all managed by a single cluster. It is possible to define up
to 8 rules.

In the current ODP release, there are some limitations:
\begin{itemize}
	\item the set of rules must be identical between all pktios for a
		given an IOETH. It means that interfaces \texttt{e0} and
		\texttt{e1} can have different set of rules, but \texttt{e0p0}
		and \texttt{e0p1} must have the same set of rules.
	\item the set of rules must be the same between all the clusters for
		a given interface.
\end{itemize}

A rule if described as follow:
\begin{lstlisting}
e0:hashpolicy=[P_{@_/_=_#_}{@_/_=_#_}...]
\end{lstlisting}
\begin{itemize}
	\item tokens \texttt{[ ]} enclose a rule. A rule is composed by an
		optional \emph{priority} and 1 to 9 \emph{rule entries}.
	\item token \texttt{P} is the priority of the current rule. It is
		optional, must appear just after the \texttt{[} token and must
		be comprised between 0 and 7. 0 is the most prioritary.
	\item tokens \texttt{\{ \}} enclose a rule entry. A rule entry is
		applied on a 64 bit word. A rule entry must be immediatly
		followed by another rule entry or by a closing rule token
		\texttt{]}. A rule entry is composed by following items:
		\begin{itemize}
			\item \emph{offset}: offset from the base of ethernet packet
				to extract the 64 bits word.
			\item \emph{comparison bytemask}: 8 bits \emph{bytemask} for
				comparing the extracted value with the extracted 64 bit
				word.
			\item \emph{comparison value}: expected 64 bit word to be
				compared with the extracted 64 bits word after applying
				comparison bytemask.
			\item \emph{hash bytemask}: 8 bits mask to select the bytes
				which will be taken into account during the hash
				computation.
		\end{itemize}
\end{itemize}

A rule will be applied for a packet if \emph{all} of its entries match
this packet. At least one hash bytemask must be specified. This hash
result will select the target cluster between all of the clusters
registered on this ethernet interface. User must ensure that there is
enough entropy with the hashed values to ensure an even dispatch between
the clusters.

If a packet does not match any defined rule, it will be dropped.

\paragraph{Examples:}

Following rule dispatch all the incoming packets on interface ethernet 0
hashing the destination MAC address:
\begin{lstlisting}
e0:hashpolicy=[{@0#0x3f}]
\end{lstlisting}

Following rule is in two times: if the incoming packet is an IPV4 packet
(\texttt{@12/0x3=0x0800}), it will be dispatched hashing the IP source
address (\texttt{@26\#0xf}). If the packet does not match the first rule
(which has the highest priority 0), it will use the second rule (with
priority 1), dispatching with the hash of the source MAC address
(\texttt{@6\#0x3f}) if the source MAC address is like
\texttt{*:77:CA:FE:*:*} (\texttt{/0x1c=0x0077cafe0000}). If the packet
does not match any of these two rules, it will be dropped.
\begin{lstlisting}
e0:hashpolicy=[P0{@12/0x3=0x0800}{@26#0xf}][P1{@6#0x3f/0x1c=0x0077cafe0000}]
\end{lstlisting}

Previous is very restrictive. It is possible to define a low priority
rule to dispatch without any match restriction, for example hashing the
destination MAC addres:
\begin{lstlisting}
e0:hashpolicy=[P0{@12/0x3=0x0800}{@26#0xf}][P1{@6#0x3f/0x1c=0x0077cafe0000}][P2{@0#0x3f}]
\end{lstlisting}


\section{Compiling and Running}

\subsection{Template Project}
A template project is available in the K1-ODP installation directory
\begin{lstlisting}
/usr/local/k1tools/share/odp/skel
\end{lstlisting}

To create a new ODP application the easiest way is to copy the
\texttt{skel} directory and edit the Makefile and source files.

In this case, building and running is easily done by running
\texttt{make APST\_DIR=/path/to/build/dir} or running the
\texttt{run.sh} in the \texttt{APST\_DIR/<app\_name>}.
By default \texttt{APST\_DIR} is the install subdirectory.


\subsection{Custom Project}

Some compiler flags must be passed to the compiler when trying to
build an ODP application.
\begin{lstlisting}
LDFLAGS += -L$($(K1_TOOLCHAIN_DIR)/lib/<port> -lodphelper -lodp -lcrypto -Wl,--gc-sections
\end{lstlisting}

It is recommended to use the pre-built firmware
\texttt{iounified.kelf} available with K1-ODP in these cases.
They can be found in
\begin{lstlisting}
/usr/local/k1tools/share/odp/firmware/<platform>/iounified.kelf
\end{lstlisting}

Running a \texttt{k1b-kalray-nodeos} application, can then be done using:
\begin{lstlisting}
k1-jtag-runner --exec-file IODDR0:/path/to/iounified.kelf --exec-file
IODDR1:/path/to/iounified.kelf --exec-file
"Cluster<X>:<executable_name>" [--exec-file
  "Cluster<Y>:<executable_name>" ...] -- <args>
\end{lstlisting}

\subsection{Simulation}

For the moment, only single cluster ODP application can be run in
simulation. In this case, only the \texttt{magic:<if\_name>} pktios
are available.

To build for simulation, either \texttt{k1b-kalray-nodeos\_simu} or
\texttt{k1b-kalray-mos\_simu} port of K1-ODP should be used.

Then, running the application can be done using this command:
\begin{lstlisting}
k1-cluster   --functional  --mboard=developer --march=bostan --user-syscall=/usr/local/k1tools/lib64/libodp_syscall.so -- <executable name> <args>
\end{lstlisting}

\section{Limitations}

\begin{itemize}
\item[-]{The current ODP version does not support ordered queues}
\item[-]{In some case where multiple Rx threads are used to handle incoming trafic
  (default), packets may not be returned in order to the ODP
  application. If order is required, the number of Rx thread must be
  set to one.}
\item[-]{It is currently not possible to do loopback between multiple
  Ethernet lane of a single MPPA board. Because odp\_pktio\_open waits
  for the link to be up before returning (with a timeout), it needs
  the opposit side of the link to be up.}
\item[-]{Only promiscuous mode is currently available.}
\item[-]{Multi-Lane (4x10G) it not supported yet.}
\end{itemize}

\end{document}
