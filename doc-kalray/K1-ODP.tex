\documentclass{trkalray}
\usepackage{listings}
\usepackage[toc,page]{appendix}
\usepackage{makeidx}
\usepackage{algorithm}
\usepackage{algorithmic}
\usepackage{caption}
\usepackage{graphicx}
\usepackage{listings}
\lstset{ %
frame=single
}

\newcommand{\MPPA}{MPPA\texttrademark\space}

\version{KETD-355}{W23}{2015}


\author{%
Kalray S.A.\autref{1}
}
\docowner{Nicolas Morey-Chaisemartin}{nmorey@kalray.eu}
\vers{1.1}
\institute{%
\autlabel{1} \email{support@kalray.eu},
Kalray S.A.
}

\abstract{%
This document describes how to use K1-ODP
}

\keywords{%
Dataflow, MPPAIPC, Examples
}

\renewcommand{\lstlistingname}{Code}% Listing -> Code


\makeindex

\title{\MPPA ODP User Guide}

\begin{document}

\maketitle

\tableofcontents

\newpage
\section{Introduction}

This documents briefly describes how to use ODP for the MPPA platform.

ODP-MPPA is currently based on ODP 1.6.

Additional generic documentation can be found on ODP website and in
\begin{lstlisting}
/usr/local/k1tools/doc/ODP/mppa/
\end{lstlisting}

\section{Requirements}

Working with K1-ODP requires two sets of packages:
\begin{itemize}
\item[-]{A complete K1 Toolchain using the k1-tools package}
\item[-]{K1-ODP Library using the k1-odp package}
\end{itemize}

The K1-ODP files will be installed under the
\texttt{/usr/local/k1tools} directory.

\section{What's new}

\subsection{K1-ODP 1.6.k7}
\begin{itemize}
\item[-]{ODP netdev driver is compatible with Centos 7.3}
\end{itemize}

\subsection{K1-ODP 1.6.k6}
\begin{itemize}
\item[-]{ODP netdev driver for x86 is available}
\item[-]{Added AB04 support}
\item[-]{ODP firmwares can support both Linux in IO and ODP in
  clusters}
\item[-]{pktio can now be tied to a specific Rx thread}
\item[-]{Add a refcount system on packets}
\item[-]{Add a pktio to receive data from IO clusters}
\item[-]{Added a default application that allows using a KONIC80 as a
  dual port netdev.}
\end{itemize}

\subsubsection{Bug Fixes}
\begin{itemize}
\item[-]{Fix packet leaks in Rx thread.}
\item[-]{Fix ethernet MAC generation when using k1-gdb}
\item[-]{Auto reinit Ethernet phy if link failed to start when using
  loopback}
\item[-]{Fix a bug in packet Tx which might cause extra data to be sent}
\end{itemize}

\subsubsection{Netdev}
A default application to use a KONIC80 is available in the
K1-odp-runtime package.

It can be started by running
\begin{lstlisting}
/usr/local/k1tools/share/odp/apps/odp-netdev/run.sh
\end{lstlisting}

Once started and if the mppapcie\_odp module is loaded, up to two netdev
interfaces should be visible from the host.
Interfaces will only show up if the associated Ethernet port has a
40Gbps cable plugged in.

These interfaces called \texttt{mpodp0.0.0.0} and
\texttt{mpodp0.0.1.0} will respectively forward packet to/from the
first and second 40G port of the KONIC80.

\subsection{K1-ODP 1.6.k5}
\begin{itemize}
\item[-]{Rx threads now use only odd PE to allow access to all crypto
  accelerators}
\item[-]{Added EMB01b support}
\end{itemize}

\subsection{K1-ODP 1.6.k4}
\begin{itemize}
\item[-]{New firmware system}
\item[-]{Remove synchronisation barrier at cluster
  initialization. \texttt{boot\_set\_nb\_clusters} should not be
  called anymore.}
\item[-]{Add non-standard \texttt{\_odp\_pktio\_stats} function to get
ethernet counters}
\item[-]{Ethernet lane is now UP when calling
  \texttt{odp\_pktio\_start} instead of open}
\item[-]{Loopback between MPPA interfaces using a cable is now
  supported. Both pktio must be created before starting them to allow
  the link to get UP.}
\item[-]{Added support for multiple ethernet lane on one I/O.}
\item[-]{Added support for EMB01b.}
\end{itemize}

\subsubsection{Firmwares}

Firmware build system and behaviour has been revised.
Although the main thread still runs on IODDR, the main does not need
to poll for message anymore.
The call to \texttt{odp\_rpc\_server\_start} starts a background
thread on IOETH that handles all the clusters request directly.
The main thread should not exit before the spawned clusters have been
joined by calling \texttt{join\_clusters}.
If no clusters were spawned, the firmare should never exit.
If the firmware is spawned uotside a MPK (no additional ELF files),
\texttt{boot\_cluster} and \texttt{boot\_clusters} do not spawn nor
parse arguments.

Firmware build system has been changed to inherit compilation flags
from the \texttt{platforms.inc} file already used for building ODP
applications and now use the same board names.

\subsubsection{Bug Fixes}
\begin{itemize}
\item[-]{Fix issues with packet ordering}
\item[-]{Fix cache coherency issue when parsing received packets}
\item[-]{Fix potential buffer overflow when pool packets are allocated
requesting less than 64B}
\end{itemize}

\subsection{K1-ODP 0.3.3}

\texttt{verbose} option has been added for Ethernet pktio.

\subsubsection{Bug Fixes}
\begin{itemize}
\item[-]{Fix packet leaked by the first \texttt{odp\_pktio\_send}}
\end{itemize}

\subsection{K1-ODP 0.3.2}
\subsubsection{Bug Fixes}
\begin{itemize}
\item[-]{Fix issue for cluster-cluster pktio when one side closes the
  interface}
\item[-]{Fix issue when opening a 40G lane from miltiple clusters}
\item[-]{Fix clean target for firmware in template project.}
\item[-]{Fix another issue in packet ordering.}
\item[-]{Fix SEGV/traps when Rx thread are under heavy loads.}
\item[-]{Add missing cache invalidation when pulling packets from a queue.}
\end{itemize}

\subsection{K1-ODP 0.3.1}

\subsubsection{Makefile.apps}

To avoid multiple builds of firmwares (developer and KONIC) the
Makefile.apps used by the TemplateProject was updated to only build
the required firmware.
Compilation should work seamlessly but the run.sh script of project
should be updated.
The new firmware is always named \texttt{firmware.kelf} and does not
contain the board type anymore.

\subsubsection{Bug Fixes}

\begin{itemize}
\item[-]{When cluster2cluster link is not yet ready, \texttt{odp\_pktio\_recv}
now returns 0 instead of -2}
\item[-]{Fix packet ordering issue, even when only on Rx thread is used.}
\end{itemize}

\newpage
\section{Choosing a version}
K1-ODP is available in multiple ports in the k1-odp package, one for
each of the platforms available for executing ODP applications.

In this version, two ports are available:
\begin{itemize}
\item[-]{\texttt{k1b-kalray-nodeos}: Bostan cluster with POSIX support}
\item[-]{\texttt{k1b-kalray-mos}: Bostan cluster directly on kalray Hypervisor}
\end{itemize}

\section{Choosing a board}

Multiple board are available. Each can run any of the ports of K1-ODP.
The available boards are:
\begin{itemize}
\item[-]{\texttt{developer}: AB01B Board with 1G, 10G and 40G (with EXB01) ports }
\item[-]{\texttt{konic80}: KONIC80 Board with 2 40G ports.}
\item[-]{\texttt{simu}: AB01B board for software simulation.}
\end{itemize}

\subsection{simu}

This ports targets one Bostan MPPA cluster in simulation.

Using the simulator, this ports allows simulated ODP applications to
have transparent access to the x86 network interfaces.

In the current version, this port supports these interfaces:
\begin{itemize}
\item[-]{loop}
\item[-]{magic:$<$if\_name$>$}
\end{itemize}

\newpage
\section{Platform Settings}
\subsection{Rx Threads}
ODP internally allocates 2 cores (Rx Threads) per cluster to convert the incoming
Ethernet, NoC, and PCIe traffic to ODP packets.

The number of Rx Threads can be selected at runtime from 1 to 6,
depending on the application requirement. Each thread can handle
approximately 1.8Mpps. Thus application with low pps requirements can
save 1 core for computation and on the other hand, high pps
application should dedicate more cores to Rx Threads to avoid packet
drop.

The number of threads can be configured by providing a
\texttt{odp\_platform\_init\_t} structure to
\texttt{odp\_init\_global} and setting the \texttt{n\_rx\_thr} value.

\begin{lstlisting}
	odp_platform_init_t platform_params = { .n_rx_thr = 1 };

	if (odp_init_global(NULL, &platform_params)) {
		fprintf(stderr, "Error: ODP global init failed.\n");
		exit(EXIT_FAILURE);
	}
\end{lstlisting}

When only one Rx thread is used, it is guaranteed that packets will be
received in order. This is automactically disabled when \texttt{n\_rx\_thr} is
larger than 1. If pktio are configured so that they only use no more
than one Rx thread each (a rx thread might still handle more than one
pktio), it is possible to force ordering to be kept by setting
\texttt{sort\_buffers} = 1 in the \texttt{odp\_platform\_init\_t} structure.


\newpage
\section{Packet IO Interfaces}

K1-ODP provides multiple pktio types that can be used to communicate
with Ethernet ports, or clusters.

\begin{itemize}
\item[-]{\texttt{loop}: Software loopback interface}
\item[-]{\texttt{e<X>}: 40G Ethernet interface of IOETH \texttt{<X>}}
\item[-]{\texttt{e<X>p<Y>}: 1/10G Ethernet interface \texttt{<Y>} of
  IOETH \texttt{<X>}}
\item[-]{\texttt{cluster<X>}: Interface to cluster \texttt{<X>}}
\item[-]{\texttt{p<X>p<Y>}: Netdev to host using \texttt{Y}th
  interface of IODDR \texttt{<X>}}
\item[-]{\texttt{magic:<if>}: Link to the x86 interface
  \texttt{<if>}. This is only available on simu platforms}
\item[-]{\texttt{ioddr<X>}: Interface to IODDR \texttt{X}. Receive only}
\end{itemize}

\subsection{Customizing Ethernet interface behavior}

It is possible to configure the way ODP handles Ethernet interfaces.
This is done by passing options through the pktio name:
\begin{lstlisting}
e0p0:<option1>:<option2>
\end{lstlisting}

\subsubsection{Rx resources}

By default Ethernet packets from a lane are distributed among 20 NoC Rx on each
cluster that opens this lane.
When working with 40G lane, or with high PPS situations, this is often
not sufficient to handle the high performance requirement.
To configure this value, the option tags can be used:
\begin{lstlisting}
e0:tags=120
\end{lstlisting}

Not that it is not possible to use more than 120 tags per pktio nor
250 tags for all pktios.

\subsubsection{Jumbo}

Jumbo frames are not allowed by default. To enable them, the option
\texttt{jumbo} must be used.
Note that the option must be the same (on or off) for all the cluster
that use an Ethernet interface

\begin{lstlisting}
e0:jumbo
\end{lstlisting}

\subsubsection{Loopback}

It is possible to put an Ethernet lane in loopback mode. In this mode,
it is not required for a cable to be plugged in, nor for the board to
have a physical connector connected to the lane.
When loopback mode is enabled, all the packets sent to this lane by
the clusters are looped back into the HW Dispatcher and sent back to
the cluster that opened this pktio.

Note that the option must be the same (on or off) for all the cluster
that use an Ethernet interface

\begin{lstlisting}
e0:loop
\end{lstlisting}

\subsubsection{Dispatch}

User can apply dispatch rules to Ethernet physical links. According to
these rules, input Ethernet packets will be dispatched to specific
clusters. It allows for example to guarantee that all packets from an IP
address are all managed by a single cluster. It is possible to define up
to 8 rules.

In the current ODP release, there are some limitations:
\begin{itemize}
	\item the set of rules must be identical between all pktios for a
		given IOETH. It means that interfaces \texttt{e0} and
		\texttt{e1} can have different set of rules, but \texttt{e0p0}
		and \texttt{e0p1} must have the same set of rules.
	\item the set of rules must be the same between all the clusters for
		a given interface.
\end{itemize}

A rule if described as follow:
\begin{lstlisting}
e0:hashpolicy=[P_{@_/_=_#_}{@_/_=_#_}...]
\end{lstlisting}
\begin{itemize}
	\item tokens \texttt{[ ]} enclose a rule. A rule is composed by an
		optional \emph{priority} and 1 to 9 \emph{fields}.
	\item token \texttt{P} is the priority of the current rule. It is
		optional, must appear just after the \texttt{[} token and must
		be comprised between 0 and 7. 0 is the most prioritized.
	\item tokens \texttt{\{ \}} enclose a field. A field is
		applied on a 64 bit word. A field must be immediately
		followed by another field or by a closing rule token
		\texttt{]}. A field is composed by following items:
		\begin{itemize}
			\item \emph{@offset}: offset from the base of Ethernet packet
				to extract the 64 bits word.
			\item \emph{/comparison bytemask}: 8 bits \emph{bytemask} for
				comparing the extracted value with the extracted 64 bit
				word.
			\item \emph{=comparison value}: expected 64 bit word to be
				compared with the extracted 64 bits word after applying
				comparison bytemask.
			\item \emph{\#hash bytemask}: 8 bits mask to select the bytes
				which will be taken into account during the hash
				computation.
		\end{itemize}
\end{itemize}

A rule will be applied for a packet if \emph{all} of its field match
with the packet. At least one hash bytemask must be specified. This hash
result will select the target cluster between all of the clusters
registered on this Ethernet interface. User must ensure that there is
enough entropy with the hashed values to ensure an even dispatch between
the clusters.

If a packet does not match any defined rule, it will be dropped.

\paragraph{Example 1:}

The following rule dispatches all the incoming packets on interface Ethernet 0
hashing the destination MAC address:
\begin{lstlisting}
e0:hashpolicy=[{@0#0x3f}]
\end{lstlisting}

\paragraph{Example 2:}

The following example has two rules:

If the incoming packet is an IPV4 packet (\texttt{@12/0x3=0x0800}),
 it will be dispatched hashing the IP source address
 (\texttt{@26\#0xf}).

If the packet does not match the first rule
(which has the highest priority 0), it will use the second rule (with
priority 1), dispatching with the hash of the source MAC address
(\texttt{@6\#0x3f}) if the source MAC address i like
\texttt{*:77:CA:FE:*:*} (\texttt{/0x1c=0x0077cafe0000}).

If the packet
does not match any of these two rules, it will be dropped.
\begin{lstlisting}
e0:hashpolicy=[P0{@12/0x3=0x0800}{@26#0xf}][P1{@6#0x3f/0x1c=0x0077cafe0000}]
\end{lstlisting}

\paragraph{Example 3:}

The previous example is very restrictive. It is possible to add a low priority
rule to dispatch without any match restriction, for example hashing the
destination MAC address:
\begin{lstlisting}
e0:hashpolicy=[P0{@12/0x3=0x0800}{@26#0xf}][P1{@6#0x3f/0x1c=0x0077cafe0000}][P2{@0#0x3f}]
\end{lstlisting}

\subsubsection{Verbose}

When activating verbose mode on an Ethernet lane, the firmware will
output messages for debugging link issues.
\begin{lstlisting}
e0:verbose
\end{lstlisting}

\newpage
\section{Compiling and Running}

\subsection{Template Project}
A template project is available in the K1-ODP installation directory
\begin{lstlisting}
/usr/local/k1tools/share/odp/skel
\end{lstlisting}

To create a new ODP application the easiest way is to copy the
\texttt{skel} directory and edit the Makefile and source files.

In this case, building and running is easily done by running
\texttt{make APST\_DIR=/path/to/build/dir} or running the
\texttt{run.sh} in the \texttt{APST\_DIR/<app\_name>}.
By default \texttt{APST\_DIR} is the install sub-directory.


\subsection{Custom Project}

Some compiler flags must be passed to the compiler when trying to
build an ODP application.
\begin{lstlisting}
LDFLAGS += -L$($(K1_TOOLCHAIN_DIR)/lib/odp/board/<port> -lodphelper -lodp -lcrypto -Wl,--gc-sections
\end{lstlisting}

It is recommended to use the pre-built firmware
\texttt{iounified.kelf} available with K1-ODP in these cases.
They can be found in
\begin{lstlisting}
/usr/local/k1tools/share/odp/firmware/<board>/<platform>/iounified.kelf
\end{lstlisting}

Running a \texttt{k1b-kalray-nodeos} application, can then be done using:
\begin{lstlisting}
k1-jtag-runner --exec-file IODDR0:/path/to/iounified.kelf --exec-file
IODDR1:/path/to/iounified.kelf --exec-file
"Cluster<X>:<executable_name>" [--exec-file
  "Cluster<Y>:<executable_name>" ...] -- <args>
\end{lstlisting}

\subsection{Simulation}

For the moment, only single cluster ODP application can be run in
simulation. In this case, only the \texttt{magic:<if\_name>} pktios
are available.

To build for simulation, \texttt{simu} board should be used.

Then, running the application can be done using this command:
\begin{lstlisting}
k1-cluster   --functional  --mboard=developer --march=bostan --user-syscall=/usr/local/k1tools/lib64/libodp_syscall.so -- <executable name> <args>
\end{lstlisting}

\section{Limitations}

\begin{itemize}
\item[-]{The current ODP version does not support ordered queues}
\item[-]{In some case where multiple Rx threads are used to handle incoming traffic
  (default), packets may not be returned in order to the ODP
  application. If order is required, the number of Rx thread must be
  set to one.}
\item[-]{Only promiscuous mode is currently available.}
\item[-]{When clusters transmits to an ethernet port more than the port
  bandwidth can handle, packets may be dropped.}
\end{itemize}

\end{document}
