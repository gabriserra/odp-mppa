\documentclass{trkalray}
\usepackage{listings}
\usepackage[toc,page]{appendix}
\usepackage{makeidx}
\usepackage{algorithm}
\usepackage{algorithmic}
\usepackage{caption}
\usepackage{graphicx}
\usepackage{listings}
\lstset{ %
frame=single
}

\newcommand{\MPPA}{MPPA\texttrademark\space}

\version{KETD-355}{W23}{2015}


\author{%
Kalray S.A.\autref{1}
}
\docowner{Nicolas Morey-Chaisemartin}{nmorey@kalray.eu}
\vers{1.1}
\institute{%
\autlabel{1} \email{support@kalray.eu},
Kalray S.A.
}

\abstract{%
This document describes how to use K1-ODP
}

\keywords{%
Dataflow, MPPAIPC, Examples
}

\renewcommand{\lstlistingname}{Code}% Listing -> Code


\makeindex

\title{K1-ODP Manual}

\begin{document}

\maketitle

\tableofcontents

\newpage
\section{Requirements}

Working with K1-ODP requires two sets of packages:
\begin{itemize}
\item[-]{A complete K1 Toolchain using the k1-tools package}
\item[-]{K1-ODP Library using the k1-odp package}
\end{itemize}

The K1-ODP files will be installed under the
\texttt{/usr/local/k1tools} directory.

\section{Choosing a version}
K1-ODP is available in multiple ports in the k1-odp package, one for
each of the platforms available for executing ODP applications.

In this version, two ports are available:
\begin{itemize}
\item[-]{\texttt{k1a-kalray-nodeos}}
\item[-]{\texttt{k1a-kalray-nodeosmagic}}
\end{itemize}

\subsection{k1a-kalray-nodeos}

This port targets one Andey MPPA cluster on hardware.

In the current version, this port supports these interfaces:
\begin{itemize}
\item[-]{\texttt{loop}: Software loopback.}
\end{itemize}

\subsection{k1a-kalray-nodeosmagic}

This ports targets one Andey MPPA cluster in simulation. It is
compiled for the large\_memory board to remove the 2MB memory limit
during the development phase.

Using the simulator, this ports allows simulated ODP applications to
have transparent access to the x86 network interfaces.

In the current version, this port supports these interfaces:
\begin{itemize}
\item[-]{\texttt{loop}: Software loopback.}
\item[-]{\texttt{magic-<ifname>}: Transparent access to the host
  \texttt{<ifname>} interface.}
\end{itemize}

\section{Compiling}

Some compiler flags must be passed to the compiler when trying to
build an ODP application.

\begin{lstlisting}
CFLAGS += -I/usr/local/k1tools/k1-nodeos/include
LDFLAGS += -L/usr/local/k1tools/k1-nodeos/lib/<port>/ -lodp
\end{lstlisting}
where port is replaced with one of the available K1-ODP port.

Note that when building with \texttt{k1a-kalray-nodeosmagic}, it is
required to use the

 \texttt{-mboard=large\_memory} flag.

\section{Running}

Running K1-ODP applications is done the same way a standard K1 single
cluster application would be run.

\subsection{k1a-kalray-nodeos}
Running a \texttt{k1a-kalray-nodeos} application, can be done either
on hardware using:
\begin{lstlisting}
k1-jtag-runner --exec-file "Cluster0:<executable name>" -- <args>
\end{lstlisting}
or in simulation using:
\begin{lstlisting}
 k1-cluster -- <executable name> <args>
\end{lstlisting}

\subsection{k1a-kalray-nodeosmagic}
Running a \texttt{k1a-kalray-nodeosmagic} application, must be done in
simulation using this command:
\begin{lstlisting}
k1-cluster   --functional  --mboard=large_memory --user-syscall=/usr/local/k1tools/lib64/libodp_syscall.so -- <executable name> <args>
\end{lstlisting}

\end{document}
